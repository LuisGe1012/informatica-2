\documentclass[10pt,a4paper]{article}
\usepackage[utf8]{inputenc}
\usepackage{amsmath}
\usepackage{amsfonts}
\usepackage{amssymb}
\usepackage{graphicx}
\begin{document}
\author{Luis Gerardo Cruz}\large
\title{Parcial 1.1}
\maketitle
\section*{Respuesta No. 1}
\begin{itemize}
\item \textbf{Nodos: } \ $ \lbrace 1,2,3,4,5,6,7 \rbrace$
\item \textbf{Vertices: }
$$
        \left\langle \left\{
            \begin{bmatrix}
                \langle 1,2 \rangle & \langle 1, 3\rangle & \langle 1,4 \rangle \\
                \rangle & \langle 1,5 \rangle & \langle 1,6 \rangle \\
                \langle 2,3 \rangle & \langle 2,4 \rangle & \langle 2,5 \rangle \\
                \langle 2,6 \rangle & \langle 3,4 \rangle & \langle 3,5 \rangle \\
                \langle 3,6 \rangle & \langle 3,7 \rangle & \langle 4,7 \rangle \\
                \langle 5,6 \rangle & \langle 5,7 \rangle & \langle 6,7
            \end{bmatrix}
        \right\}  \right\rangle
    $$ 
\item \textbf{Grafo: }
\begin{center}
\includegraphics[width=18cm]{Untitled.png}  
\end{center}
\end{itemize}
\section*{Respuesta No. 2}
\begin{center}
$ \sum_{i=1}^{n}{i}=\frac{n(n+1)}{2} $
\end{center}
\begin{itemize}
\item\textbf{Caso Base: } \ $ N = 1 $
\end{itemize}

\begin{center}
\
\\ $ 1 = \frac{1(1+1)}{2} $
\
\\ $ 1 = \frac{1(2)}{2}$
\
\\ $ 1 = \frac{2}{2}$
\
\\ $ 1 = 1 $
\end{center}
\begin{itemize}
\item\textbf{Caso inductivo: } \ $ \forall n $
\item\textbf{Hipotesis inductiva: } \ supongamos entonces que p(n) es verdadera, es decir que 1+2+3+...+n=n(n+1)/2 es verdadera
\begin{center}

$ \sum_{i=1}^{n}{i}=\frac{n(n+1)}{2} $
\end{center}
\item\textbf{Demostracion: n = n+1}
\\
\\
$ \sum_{i=1}^{n}{i}= \frac{n+1(n+1+1)}{2} $
$					= \frac{n+1(n+2)}{2} $
$					= \frac{n+1 [(n+1)+1]}{2} $
\
\end{itemize}
\section*{Respuesta No. 3}
\begin{center}
$ \sum_{i=1}^{n}{1} \ =1+2+3+4+\ \ldots\ +n $

\
\\ $ \sum(n)   \left\{
                        \begin{array}{ll}
                                0  & \mbox{si } n = 0 \\
                                (n-i+1) & \mbox{si } n = s(a) \ y \ i=s(b)
                        \end{array}
                \right.
$
\end{center}
\section*{Respuesta No. 4}
\begin{center}
$ a\oplus b = b\oplus a $
\end{center}
\begin{itemize}
\item\textbf{Caso base: } \ $ a=0 $
\begin{center}
 $ 0\oplus b = b\oplus 0$
  \
  \\ $ b=b $
\end{center}
\item\text{b =  0}
\begin{center}
$ a\oplus 0 = 0\oplus a$
  \
  \\ $ a=a $
\end{center}
\item\textbf{Caso inductivo: }
\\
\\
$ a\oplus b = b\oplus a $
\\
\item\textbf{Demostracion}
\begin{center}
$ s(a)\oplus b = b\oplus s(a) $
\\
$ s(a\oplus b) = b\oplus s(a) $
\\
$ s(a\oplus b) = s(b\oplus a) $
\end{center}
\end{itemize}
\section*{Repuesta No. 5}

Dada la funci\'on $a\geq b$ para numeros naturales unarios:
\[
        a\geq b =
                \left\{
                        \begin{array}{ll}
                                s(o)  & \mbox{si } b = o \\
                                o & \mbox{si } a = o \\
                                i\geq j & \mbox{si } a = s(i)\ \&\ b = s(j)
                        \end{array}
                \right.
\]
\begin{itemize}
\item\textbf{Caso base: } n = 0 
\begin{center}
$ (0+0) \geq s(0)  $
\end{center}
\item\textbf{Caso inductivo: }
\begin{center}
$ s(i)\oplus s(i) \geq s(i) $
\\
$ s(s(i)\geq s(i) $
\\
$ s(s(i)\ominus s(i)\geq 0 $
\\
$ s(i)\geq 0 $
\\
\
$ s(i)\geq 0 = s(0) $
\end{center}
\end{itemize}

\end{document}
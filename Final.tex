\documentclass[10pt,a4paper]{article}
\usepackage[utf8]{inputenc}
\usepackage{amsmath}
\usepackage{amsfonts}
\usepackage{amssymb}
\begin{document}                          
\title{Hoja de Trabajo No.1}\normalsize
\author{Luis Gerardo Cruz}
\maketitle
\section*{Ejercicio 2 }
\subsection{conjunto de nodos}
\{1,2,3,4,5,6\}
\subsection{Conjunto de vertices }

    $$
        \left\langle \left\{
            \begin{bmatrix}
                \langle 1,2 \rangle & \langle 1, 3\rangle & \langle 1,4 \rangle \\
                \langle 1,5 \rangle & \langle 2,3 \rangle & \langle 2,4 \rangle \\
                \langle 2,6 \rangle & \langle 3,6 \rangle & \langle 3,5 \rangle \\
                \langle 4,6 \rangle & \langle 4,5 \rangle & \langle 5,6 \rangle \\
           
            \end{bmatrix}
        \right\}, 1, 6 \right\rangle
    $$ \\
\section*{Ejercicio 3}
\subsection{>Que estructura de datos podria representar un lanzamiento de dados?}
\ R// Una estructura de camino 
\subsection{>Que algoritmo podriamos utilizar para generar dicha estructura?}
\ R// Se usario un algoritmo que al primer lazamiento siempre nos de el lado "1" como resultado, y los siguientes lanzamientos nos daran un resultado aleatorio con una probabilidad de 1/6 para cada lado de nuestro dado 
\subsection{>Como nos aseguramos que ese algoritmo siempre produce un resultado?}
\ R// Determinando sus 6 distintos lados como posbibles respuestas y que siempre que el dado caiga nos de como resultado una de sus caras de forma aleatoria 
\end{document}

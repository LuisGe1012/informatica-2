\documentclass[10pt,a4paper]{article}
\usepackage[utf8]{inputenc}
\usepackage{amsmath}
\usepackage{amsfonts}
\usepackage{amssymb}
\renewcommand{\baselinestretch}{1.5}
% Margins
\topmargin=-0.45in
\evensidemargin=0in
\oddsidemargin=0in
\textwidth=6.5in
\textheight=9.0in
\headsep=0.25in
\linespread{1.1} 
\begin{document}
\title{Hoja de trabajo No.2}\large
\author{Luis Gerardo Cruz}
\maketitle
\section*{Ejercicio 1}
\begin{large}
\textbf{Demostrar utilizando induccion}
\end{large} 
\ \[
        \forall\ n.\ n^3\geq n^2
\]
\\donde $n\in\mathbb{N}$

\
\\\textbf{Solucion:}

\ \begin{flushleft}
• Caso base: n = 0
\end{flushleft}

\[
0^{3}\geq0^{2}= 0\geq0
\]
\ \begin{flushleft}
• Caso inductivo: $n\in\mathbb{N}$
\end{flushleft}

\ Hipotesis inductiva: \[
        \ n^3\geq n^2 \ = n*(n^2)\geq (n^2)
\]

\ Sucesor: S(n) = (n+1) 

\
\\\textbf{Demostracion:}

\
\\ $ (n+1)(n+1)^2\geq (n+1)^2 $ 
\
\\ $ (n+1)\geq (n+1)^2/(n+1)^2 $
\
\\ $ (n+1)\geq 1$
\
\\ $ n+1\geq 1 $
\
\\ $ n\geq 1-1 $
\
\\ $ n\geq 0 $ 

\
\\ A pesar de que la demostracion no llega a la hipotesis inductiva, en esta misma se puede observar que se cumple el parametro requerido en el cual n cumple la funcion de ser mayor o igual en todos los casos, ya que sea n cualquier $ \mathbb{N}$, siempre sera mayor o igual. 
\section*{Ejercicio2}
\textbf{Demostrar utilizando induccion la desigualdad de Bernoulli:}
\[
        \forall\ n.\ (1+x)^n\geq nx
\]
\\donde $n\in \mathbb{N}$, $x\in \mathbb{Q}$ y $x\geq -1$
\\
\\
\textbf{Solucion:} 
\begin{itemize}
\item Caso base: n = 0
\\
\\
\ cuando n es igual a 0, entonces:
\[
\ (1+x)^0\geq 0x\] 
\[\ 1\geq 0\]

\item Caso Inductivo: $n\in \mathbb{N}$, $x\in \mathbb{Q}$ y $x\geq -1$
\\
\\
\ hipotesis inductiva: $n.\ (1+x)^n\geq nx+1$
\\
\item Demostracion:
\\
\ $(1+x)^n\geq nx+1$ 
\\
\ $(1+x)^n (1+x)^1\geq (nx+1)(1+x)$
\\
\ $(1+x)^{n+1}\geq nx+nx^2+1+x$ 
\\
\\
\ Si: $nx^2\geq 0$ Entonces: $(1+x)^{n+1}\geq nx+x+1$
\\
\ $(1+x)^{n+1}\geq x(n+1)+1$
\\
\\
\ Con esto se demuestra que para cualquier n o x la desigualdad siemopre va terminar siendo mayor o igual que tal como estipula 
\end{itemize}
\end{document}